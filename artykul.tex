\documentclass[a4paper,twocolumn,11pt]{article}
\usepackage{polski}
\usepackage{fontspec}
\usepackage{xunicode}
\usepackage{xltxtra}
\defaultfontfeatures{Mapping=tex-text}
\usepackage{fullpage}
\usepackage{hyperref}
\setmainfont{Nimbus Roman No9 L}
\setmonofont{Anonymous Pro}

\title{\LARGE\bf Prosty szablon \LaTeX{} na cele SDM-2.B}
\author{\Large Jan Stępień
\\ \Large Institute of Computer Science, Warsaw University of Technology,
\\ \Large ul. Nowowiejska 15/19, 00-665 Warsaw, Poland
\normalsize
\\ jan@stepien.cc}
\date{}

% numerowanie sekcji i podseckcji
% umożliwia wstawianie \label dla sekcji, które pozwalają
% na odwoływanie się do punktów przy pomocy \ref.
\newcounter{sectioncounter} \setcounter{sectioncounter}{0} % create a counter and set it to 0
\newcounter{subsectioncounter}[sectioncounter] \setcounter{subsectioncounter}{0} % a counter subsectioncounter is reset to zero every time sectioncounter is increased
\newcommand\myabstract[1]{\hspace{-\parindent}\textbf{#1.}\hspace{6pt}}
\newcommand\mysec[1]{\refstepcounter{sectioncounter}\hspace{-\parindent}\textbf{\arabic{sectioncounter}. #1.}\hspace{6pt}}
\newcommand\mysubsec[1]{\refstepcounter{subsectioncounter}\hspace{-\parindent}\textbf{\arabic{sectioncounter}.\arabic{subsectioncounter}. #1.}\hspace{3pt}}

\begin{document}
\maketitle

\myabstract{Streszczenie}
{
\it
Niniejszy plik stanowi prościutki szablon na potrzeby tworzenia artykułów
w~ramach SDM-2.B.
Szablon objęty jest prawami autorskimi.
Postanowienia licencyjne przedstawione są w~punkcie 1.
W punkcie 2. zawarto garść szczegółów technicznych.
}

\mysec{Licencja}
\label{refmysec:licencja}
Kod źródłowy, który posłużył do wygenerowania niniejszego pliku, może być
wykorzystywany jedynie zgodnie z~postanowieniami poniższej licencji, zwanej
dalej Licencją Pocztówkową.

\mysubsec{Licencja pocztówkowa}
Pliki opublikowane na warunkach Licencji Pocztówkowej mogą być swobodnie
rozpowszechniane, wykorzystywane i~dowolnie modyfikowane, zarówno w~celach
akademickich, prywatnych jak i~komercyjnych, bez uiszczania żadnych opłat
licencyjnych, pod warunkiem spełnienia wszystkich poniższych warunków.

\begin{enumerate}
  \item
  Rozpowszechniając pliki opublikowane na zasadach Licencji Pocztówkowej
  w~postaci źródłowej, zmodyfikowane lub zachowane w postaci pierwotnej,
  licencjobiorca zobowiązuje się dołączyć do nich pełen tekst niniejszej
  licencji i~wyraźnie zaznaczyć, że rozpowszechniany plik został opublikowany na
  jej warunkach.

  \item
  Korzystając z pliku opublikowanego na zasadach niniejszej licencji
  licencjobiorca zobowiązuje się na wysłanie pierwotnemu licencjodawcy ładnej
  pocztówki ze swojej zagranicznej podróży na bieżący adres pierwotnego
  licencjodawcy. Bieżący adres należy uzyskać wysyłając stosowne zapytanie na
  adres poczty elektronicznej jan@stepien.cc .  Pocztówka powinna zostać nadana
  nie później niż w trakcie trzeciej zagranicznej podróży licencjobiorcy.

  \item
  Wszelkie niejasności wynikające z zastosowania warunków niniejszej licencji
  będą rozwiązywane polubownie.

  \item
  W~wypadku niemożności dojścia do porozumienia na drodze polubownej, spór
  zostanie rozstrzygnięty na drodze sądowej w sądzie właściwym dla miejsca
  zamieszkania pierwotnego licencjodawcy.
\end{enumerate}

\mysubsec{Postanowienia końcowe}
W przypadku korzystania z numerowania sekcji i podsekcji, pocztówka może zostać wysłana do Adama (na uprzednio zdefiniowanych zasadach), którego adres korespondencyjny można uzyskać wysyłając wiadomość na adres adam.dziedzi@gmail.com.

\mysec{Sprawy techniczne}
Do kompilacji należy użyć programu \texttt{xelatex}.
Jeżeli zainstalowana dystrybucja \LaTeX{}a~\cite{latex} nie zawiera tego
programu, mamy problem.
\texttt{xelatex} pozwala na proste wykorzystywanie czcionek w~formatach TTF i~OTF.
Bez wsparcia czcionek tych typów spełnienie wymagania dotyczącego użycia Times
New Roman może wymagać odrobiny gimnastyki.

W~niniejszym artykule użyto czcionki Nimbus Roman No9~L, która stanowi niemal
identyczną alternatywę dla Times New Roman.
Zmiany można dokonać przy pomocy modyfikacji komendy \verb|\setmainfont|.

Można łatwo przejść do poprzedniej sekcji używając tej referencji~\ref{refmysec:licencja}.

\paragraph{Bibliografia}

\bibliographystyle{plplainurl}
\renewcommand\section[2]{}
\bibliography{bibliografia}

\end{document}
